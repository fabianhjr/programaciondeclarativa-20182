\PassOptionsToPackage{unicode=true}{hyperref} % options for packages loaded elsewhere
\PassOptionsToPackage{hyphens}{url}
%
\documentclass[spanish,]{article}
\usepackage{lmodern}
\usepackage{amssymb,amsmath}
\usepackage{ifxetex,ifluatex}
\usepackage{fixltx2e} % provides \textsubscript
\ifnum 0\ifxetex 1\fi\ifluatex 1\fi=0 % if pdftex
  \usepackage[T1]{fontenc}
  \usepackage[utf8]{inputenc}
  \usepackage{textcomp} % provides euro and other symbols
\else % if luatex or xelatex
  \usepackage{unicode-math}
  \defaultfontfeatures{Ligatures=TeX,Scale=MatchLowercase}
\fi
% use upquote if available, for straight quotes in verbatim environments
\IfFileExists{upquote.sty}{\usepackage{upquote}}{}
% use microtype if available
\IfFileExists{microtype.sty}{%
\usepackage[]{microtype}
\UseMicrotypeSet[protrusion]{basicmath} % disable protrusion for tt fonts
}{}
\IfFileExists{parskip.sty}{%
\usepackage{parskip}
}{% else
\setlength{\parindent}{0pt}
\setlength{\parskip}{6pt plus 2pt minus 1pt}
}
\usepackage{hyperref}
\hypersetup{
            pdftitle={Tarea 0},
            pdfauthor={Fabián Heredia Montiel},
            pdfborder={0 0 0},
            breaklinks=true}
\urlstyle{same}  % don't use monospace font for urls
\usepackage{color}
\usepackage{fancyvrb}
\newcommand{\VerbBar}{|}
\newcommand{\VERB}{\Verb[commandchars=\\\{\}]}
\DefineVerbatimEnvironment{Highlighting}{Verbatim}{commandchars=\\\{\}}
% Add ',fontsize=\small' for more characters per line
\newenvironment{Shaded}{}{}
\newcommand{\AlertTok}[1]{\textcolor[rgb]{1.00,0.00,0.00}{\textbf{#1}}}
\newcommand{\AnnotationTok}[1]{\textcolor[rgb]{0.38,0.63,0.69}{\textbf{\textit{#1}}}}
\newcommand{\AttributeTok}[1]{\textcolor[rgb]{0.49,0.56,0.16}{#1}}
\newcommand{\BaseNTok}[1]{\textcolor[rgb]{0.25,0.63,0.44}{#1}}
\newcommand{\BuiltInTok}[1]{#1}
\newcommand{\CharTok}[1]{\textcolor[rgb]{0.25,0.44,0.63}{#1}}
\newcommand{\CommentTok}[1]{\textcolor[rgb]{0.38,0.63,0.69}{\textit{#1}}}
\newcommand{\CommentVarTok}[1]{\textcolor[rgb]{0.38,0.63,0.69}{\textbf{\textit{#1}}}}
\newcommand{\ConstantTok}[1]{\textcolor[rgb]{0.53,0.00,0.00}{#1}}
\newcommand{\ControlFlowTok}[1]{\textcolor[rgb]{0.00,0.44,0.13}{\textbf{#1}}}
\newcommand{\DataTypeTok}[1]{\textcolor[rgb]{0.56,0.13,0.00}{#1}}
\newcommand{\DecValTok}[1]{\textcolor[rgb]{0.25,0.63,0.44}{#1}}
\newcommand{\DocumentationTok}[1]{\textcolor[rgb]{0.73,0.13,0.13}{\textit{#1}}}
\newcommand{\ErrorTok}[1]{\textcolor[rgb]{1.00,0.00,0.00}{\textbf{#1}}}
\newcommand{\ExtensionTok}[1]{#1}
\newcommand{\FloatTok}[1]{\textcolor[rgb]{0.25,0.63,0.44}{#1}}
\newcommand{\FunctionTok}[1]{\textcolor[rgb]{0.02,0.16,0.49}{#1}}
\newcommand{\ImportTok}[1]{#1}
\newcommand{\InformationTok}[1]{\textcolor[rgb]{0.38,0.63,0.69}{\textbf{\textit{#1}}}}
\newcommand{\KeywordTok}[1]{\textcolor[rgb]{0.00,0.44,0.13}{\textbf{#1}}}
\newcommand{\NormalTok}[1]{#1}
\newcommand{\OperatorTok}[1]{\textcolor[rgb]{0.40,0.40,0.40}{#1}}
\newcommand{\OtherTok}[1]{\textcolor[rgb]{0.00,0.44,0.13}{#1}}
\newcommand{\PreprocessorTok}[1]{\textcolor[rgb]{0.74,0.48,0.00}{#1}}
\newcommand{\RegionMarkerTok}[1]{#1}
\newcommand{\SpecialCharTok}[1]{\textcolor[rgb]{0.25,0.44,0.63}{#1}}
\newcommand{\SpecialStringTok}[1]{\textcolor[rgb]{0.73,0.40,0.53}{#1}}
\newcommand{\StringTok}[1]{\textcolor[rgb]{0.25,0.44,0.63}{#1}}
\newcommand{\VariableTok}[1]{\textcolor[rgb]{0.10,0.09,0.49}{#1}}
\newcommand{\VerbatimStringTok}[1]{\textcolor[rgb]{0.25,0.44,0.63}{#1}}
\newcommand{\WarningTok}[1]{\textcolor[rgb]{0.38,0.63,0.69}{\textbf{\textit{#1}}}}
\setlength{\emergencystretch}{3em}  % prevent overfull lines
\providecommand{\tightlist}{%
  \setlength{\itemsep}{0pt}\setlength{\parskip}{0pt}}
\setcounter{secnumdepth}{0}
% Redefines (sub)paragraphs to behave more like sections
\ifx\paragraph\undefined\else
\let\oldparagraph\paragraph
\renewcommand{\paragraph}[1]{\oldparagraph{#1}\mbox{}}
\fi
\ifx\subparagraph\undefined\else
\let\oldsubparagraph\subparagraph
\renewcommand{\subparagraph}[1]{\oldsubparagraph{#1}\mbox{}}
\fi

% set default figure placement to htbp
\makeatletter
\def\fps@figure{htbp}
\makeatother

\ifnum 0\ifxetex 1\fi\ifluatex 1\fi=0 % if pdftex
  \usepackage[shorthands=off,main=spanish]{babel}
\else
  % load polyglossia as late as possible as it *could* call bidi if RTL lang (e.g. Hebrew or Arabic)
  \usepackage{polyglossia}
  \setmainlanguage[]{spanish}
\fi

\title{Tarea 0}
\author{Fabián Heredia Montiel}
\date{}

\begin{document}
\maketitle

\hypertarget{suma-naturales}{%
\section{1. Suma Naturales}\label{suma-naturales}}

\hypertarget{a.-suma-cero}{%
\subsection{a. Suma Cero}\label{a.-suma-cero}}

Pd. \(\forall n: \mathbb{N} [n + 0 = n]\)

\textbf{Base Inductiva}

\(0 + 0 \overset{\text{def}}=0\)

\textbf{Hipótesis Inductiva}

\(n + 0 = n\)

\textbf{Paso Inductivo}

\(Sn + 0 \overset{\text{def}}=S(n + 0) \overset{\text{hip}}=Sn\)

\hypertarget{b.-suma-uno-es-sucesor}{%
\subsection{b. Suma uno es Sucesor}\label{b.-suma-uno-es-sucesor}}

Pd. \(\forall n: \mathbb{N} [n + 1 = Sn]\)

\textbf{Base Inductiva}

\(0 + 1 \overset{\text{def}}=1 \overset{\text{def}}=S 0\)

\textbf{Hipótesis Inductiva}

\(n + 1 = Sn\)

\textbf{Paso Inductivo}

\(Sn + 1 \overset{\text{def}}=S(n + 1) \overset{\text{hip}}=S(Sn)\)

\hypertarget{multiplicaciuxf3n-de-naturales}{%
\section{2. Multiplicación de
Naturales}\label{multiplicaciuxf3n-de-naturales}}

\(n*0 := 0\)

\(n*Sm := n + (n*m)\)

\hypertarget{a.-cero-por-algo-es-cero}{%
\subsection{a. Cero por Algo es Cero}\label{a.-cero-por-algo-es-cero}}

Pd. \(\forall n : \mathbb{N} [0*n = 0]\)

\textbf{Base Inductiva}

\(0*0 \overset{\text{def}}=0\)

\textbf{Hipótesis Inductiva}

\(0*n = 0\)

\textbf{Paso Inductivo}

\(0*Sn \overset{\text{def}}=0 + (0*n) \overset{\text{hip}}=0 + 0 \overset{\text{def}}=0\)

\hypertarget{b.-uno-es-identidad-derecha}{%
\subsection{b. Uno es identidad
derecha}\label{b.-uno-es-identidad-derecha}}

Pd. \(\forall n : \mathbb{N} [n*1 = n]\)

\(n*1 \overset{\text{def}}=n*(S 0) \overset{\text{def}}=n + (n * 0) \overset{\text{def}}=n + 0 = n\)

\hypertarget{igualdad}{%
\section{3. Igualdad}\label{igualdad}}

\(f(0,0) := \top\)

\(f(Sn,Sm) := f(n,m)\)

\(f(n,m) := \bot\)

\hypertarget{funciuxf3n-misteriosa}{%
\section{4. Función Misteriosa}\label{funciuxf3n-misteriosa}}

\[
h(n,m) =
    \begin{cases}
        1           & \text{si} \ m = 0 \\
        h(n*n, m/2) & \text{si} \ m = 2m' \\
        n*h(n, m-1) & \text{si} \ m = 2m' + 1 \\
    \end{cases}
\]

Renombrando y usando notación infija

\[
n^m =
    \begin{cases}
        1           & \text{si} \ m = 0 \\
        (n*n)^{m/2} & \text{si} \ m = 2m' \\
        n*(n^{m-1}) & \text{si} \ m = 2m' + 1 \\
    \end{cases}
\]

La función misteriosa es la exponencial.

\hypertarget{lista}{%
\section{5. Lista}\label{lista}}

\hypertarget{a.-concatenar-lista-vacuxeda-es-neutro}{%
\subsection{a. Concatenar Lista Vacía es
Neutro}\label{a.-concatenar-lista-vacuxeda-es-neutro}}

Pd. \(\forall l: l_A [l ++ [] = l**\)

\textbf{Base Inductiva}

\([] ++ [] \overset{\text{def}}=[]\)

\textbf{Hipótesis Inductiva}

\(l ++ [] = l\)

\textbf{Paso Inductivo}

\((x:l) ++ [] \overset{\text{def}}=x:(l ++ []) \overset{\text{hip}}=x:l\)

\hypertarget{reversa-de-lista}{%
\section{6. Reversa de Lista}\label{reversa-de-lista}}

\hypertarget{a.-definiciuxf3n-recursiva}{%
\subsection{a. Definición Recursiva}\label{a.-definiciuxf3n-recursiva}}

\(rev [] = []\)

\(rev (x:xs) = (rev xs) ++ [x]\)

\hypertarget{operaciones-de-naturales}{%
\section{7. Operaciones de Naturales}\label{operaciones-de-naturales}}

\begin{Shaded}
\begin{Highlighting}[]
\OtherTok{suma ::} \DataTypeTok{Nat} \OtherTok{->} \DataTypeTok{Nat} \OtherTok{->} \DataTypeTok{Nat}
\NormalTok{suma n }\DataTypeTok{Cero}  \FunctionTok{=}\NormalTok{ n}
\NormalTok{suma n (}\DataTypeTok{S}\NormalTok{ m) }\FunctionTok{=} \DataTypeTok{S}\NormalTok{(suma n m)}

\OtherTok{prod ::} \DataTypeTok{Nat} \OtherTok{->} \DataTypeTok{Nat} \OtherTok{->} \DataTypeTok{Nat}
\NormalTok{prod _ }\DataTypeTok{Cero}  \FunctionTok{=} \DataTypeTok{Cero}
\NormalTok{prod n (}\DataTypeTok{S}\NormalTok{ m) }\FunctionTok{=}\NormalTok{ suma n (prod n m)}

\OtherTok{iguales ::} \DataTypeTok{Nat} \OtherTok{->} \DataTypeTok{Nat} \OtherTok{->} \DataTypeTok{Bool}
\NormalTok{iguales }\DataTypeTok{Cero}  \DataTypeTok{Cero}  \FunctionTok{=} \DataTypeTok{True}
\NormalTok{iguales (}\DataTypeTok{S}\NormalTok{ n) (}\DataTypeTok{S}\NormalTok{ m) }\FunctionTok{=}\NormalTok{ iguales n m}
\NormalTok{iguales    _     _  }\FunctionTok{=} \DataTypeTok{False}
\end{Highlighting}
\end{Shaded}

\hypertarget{exponencial}{%
\section{8. Exponencial}\label{exponencial}}

\begin{Shaded}
\begin{Highlighting}[]
\OtherTok{par ::} \DataTypeTok{Nat} \OtherTok{->} \DataTypeTok{Bool}
\NormalTok{par }\DataTypeTok{Cero} \FunctionTok{=} \DataTypeTok{True}
\NormalTok{par (}\DataTypeTok{S}\NormalTok{ m) }\FunctionTok{=}\NormalTok{ no }\FunctionTok{.}\NormalTok{ par }\FunctionTok{$}\NormalTok{ m}

\OtherTok{mitad ::} \DataTypeTok{Nat} \OtherTok{->} \DataTypeTok{Nat}
\NormalTok{mitad }\DataTypeTok{Cero} \FunctionTok{=} \DataTypeTok{Cero}
\NormalTok{mitad (}\DataTypeTok{S}\NormalTok{ (}\DataTypeTok{S}\NormalTok{ n)) }\FunctionTok{=} \DataTypeTok{S}\NormalTok{ (mitad n)}
\NormalTok{mitad (}\DataTypeTok{S}\NormalTok{ _) }\FunctionTok{=} \DataTypeTok{Cero}

\OtherTok{h ::} \DataTypeTok{Nat} \OtherTok{->} \DataTypeTok{Nat} \OtherTok{->} \DataTypeTok{Nat}
\NormalTok{h _ }\DataTypeTok{Cero} \FunctionTok{=} \DataTypeTok{S} \DataTypeTok{Cero}
\NormalTok{h n (}\DataTypeTok{S}\NormalTok{ m)}
  \FunctionTok{|}\NormalTok{ par (}\DataTypeTok{S}\NormalTok{ m) }\FunctionTok{=}\NormalTok{ h (prod n n) (mitad (}\DataTypeTok{S}\NormalTok{ m))}
  \FunctionTok{|}\NormalTok{ otherwise }\FunctionTok{=}\NormalTok{ prod n (h n m)}
\end{Highlighting}
\end{Shaded}

\hypertarget{primero-y-ultimo}{%
\section{9. Primero y Ultimo}\label{primero-y-ultimo}}

\begin{Shaded}
\begin{Highlighting}[]
\OtherTok{primero ::}\NormalTok{ [a] }\OtherTok{->}\NormalTok{ a}
\NormalTok{primero []     }\FunctionTok{=}\NormalTok{ undefined}
\NormalTok{primero (x}\FunctionTok{:}\NormalTok{_) }\FunctionTok{=}\NormalTok{ x}

\OtherTok{ultimo ::}\NormalTok{ [a] }\OtherTok{->}\NormalTok{ a}
\NormalTok{ultimo     [] }\FunctionTok{=}\NormalTok{ undefined}
\NormalTok{ultimo    [x] }\FunctionTok{=}\NormalTok{ x}
\NormalTok{ultimo (_}\FunctionTok{:}\NormalTok{xs) }\FunctionTok{=}\NormalTok{ ultimo xs}

\OtherTok{pyu ::}\NormalTok{ [a] }\OtherTok{->}\NormalTok{ (a,a)}
\NormalTok{pyu l }\FunctionTok{=}\NormalTok{ (primero l, ultimo l)}
\end{Highlighting}
\end{Shaded}

\hypertarget{clona}{%
\section{10. Clona}\label{clona}}

\begin{Shaded}
\begin{Highlighting}[]
\OtherTok{clona ::}\NormalTok{ [}\DataTypeTok{Int}\NormalTok{] }\OtherTok{->}\NormalTok{ [}\DataTypeTok{Int}\NormalTok{]}
\NormalTok{clona [] }\FunctionTok{=}\NormalTok{ []}
\NormalTok{clona (x}\FunctionTok{:}\NormalTok{xs) }\FunctionTok{=}\NormalTok{ aux x }\FunctionTok{++}\NormalTok{ clona xs}
  \KeywordTok{where}\NormalTok{ aux n }\FunctionTok{=}\NormalTok{ [n }\FunctionTok{|}\NormalTok{ _ }\OtherTok{<-}\NormalTok{ [}\DecValTok{1}\FunctionTok{..}\NormalTok{n]]}
\end{Highlighting}
\end{Shaded}

\hypertarget{agrupa}{%
\section{11. Agrupa}\label{agrupa}}

\begin{Shaded}
\begin{Highlighting}[]
\OtherTok{agrupa ::}\NormalTok{ [}\DataTypeTok{Int}\NormalTok{] }\OtherTok{->}\NormalTok{ [[}\DataTypeTok{Int}\NormalTok{]]}
\NormalTok{agrupa []     }\FunctionTok{=}\NormalTok{ []}
\NormalTok{agrupa (x}\FunctionTok{:}\NormalTok{xs) }\FunctionTok{=}\NormalTok{ (x}\FunctionTok{:}\NormalTok{grupo) }\FunctionTok{:}\NormalTok{ agrupa restante}
  \KeywordTok{where}
\NormalTok{    repetido y }\FunctionTok{=}\NormalTok{ x }\FunctionTok{==}\NormalTok{ y}
\NormalTok{    grupo      }\FunctionTok{=}\NormalTok{ takeWhile repetido xs}
\NormalTok{    restante   }\FunctionTok{=}\NormalTok{ dropWhile repetido xs}
\end{Highlighting}
\end{Shaded}

\hypertarget{freq}{%
\section{12. Freq}\label{freq}}

\begin{Shaded}
\begin{Highlighting}[]
\OtherTok{freq ::}\NormalTok{ [}\DataTypeTok{Int}\NormalTok{] }\OtherTok{->}\NormalTok{ [(}\DataTypeTok{Int}\NormalTok{, }\DataTypeTok{Int}\NormalTok{)]}
\NormalTok{freq    []  }\FunctionTok{=}\NormalTok{ []}
\NormalTok{freq (x}\FunctionTok{:}\NormalTok{xs) }\FunctionTok{=}\NormalTok{ (x, length mismos) }\FunctionTok{:}\NormalTok{ freq restantes}
  \KeywordTok{where}
\NormalTok{    mismos    }\FunctionTok{=}\NormalTok{ filter (x }\FunctionTok{==}\NormalTok{) (x}\FunctionTok{:}\NormalTok{xs)}
\NormalTok{    restantes }\FunctionTok{=}\NormalTok{ filter (x }\FunctionTok{/=}\NormalTok{)    xs}
\end{Highlighting}
\end{Shaded}

\end{document}
